In concluding our evaluation in this paper, we note that this paper
confirmed many previous observations of related work: saturation
effects~\cite{Romero2011hashtag}, importance of directionality of
interactions~\cite{saez2011high} as well as real vs. virtual
interactions~\cite{brandtzag2011facebook}.  This work also elucidated
some fascinating new insights.  For one, saturation effects in
exposure curves as noted in~\cite{Romero2011hashtag} appear to be
social \emph{interaction-dependent}.  Secondly, certain interactions
are highly predictive of certain \textit{Like Types} and certain
specific \emph{Interaction Directions}, \emph{Modalitities}, and
\emph{Actions} are far more predictive of general likes (i.e.,
outgoing photo likes) than others.  Finally, common user traits such
as \emph{Interests} and \emph{History} display interesting and unusual
trends w.r.t.\ exposure and group size that indicate different social
mechanisms may be at work behind some of these traits --- an important
topic for future exploration.

Perhaps most importantly though, comparing the overall predictiveness
of user social \textit{Interactions} vs. user traits such as
\textit{Demographics} or \textit{Interests} and \textit{History}
taking into account group size, we note that in general, the most
predictive (and statistically significant) social interactions far
outperformed any of the user traits, indicating that social groups
defined by interactions may form the strongest features from which to
form predictors for preferences in a social setting.
