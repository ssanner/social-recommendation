\label{sec:introduction}

Social networks such as Facebook record an extremely rich set of user
preferences (likes of links, posts, photos, videos) and user traits
and interactions (conversation streams, tagging, group memberships,
interests, personal history and demographic data).  In the context of
both recent work on social recommendation~\cite{sorec,ste,lla} and
information diffusion in social network analysis
~\cite{lerman2010information,Romero2011hashtag,Bakshy2012chamber}, 
it is important to know which of these interactions or common traits
are actually reflective of common preferences.

To evaluate this question, we have built a Facebook App to collect a
large subset of user information available through the Facebook App
API for over 100 users who have agreed to use this App for over a 4
month period.  This has provided us with access to full interactions
for these App users with their over 39,000+ friends which allows us to
perform a fine-grained analysis of what user interactions
and user interests are predictive of preferences 
expressed as ``likes'' on Facebook.  For example, in this paper,
we provide an empirical analysis for questions such as the following:
\begin{itemize}
\item What is the probability that a random user $u$ will like
photos that have been liked at least $k$ times by friends whose
posts $u$ has commented on?
\item What is the probability that a random user $u$ will like any
link liked at least $k=2$ times by friends sharing common interests
(movies, television, etc.) or history (same school or employer) where
the maximum number of friends sharing this interest is at most $n$?
\item What is the probability that a random male user $u$ 
will like items liked at least $k=1$ time by his female friends?
\end{itemize}

While many of these queries may seem very narrow, we note that subtle
changes in the query conditions and parameters above can lead to
drastic swings in predictive probability.  For example, in the above
queries, large variations in probability may be 
observed by changing photo likes to post likes, changing
wall comment interactions to photo comments, changing from $k=1$ to
$k=2$ or $n=2$ to $n=4$, and changing the group definition from common
school to common employer.

To better understand these subleties and to understand what what
social interactions and user traits reflect common preferences on
Facebook, we proceed in the following sections to describe our data,
our experimental methodology, and various analyses according to our
methodology that shed light on the above questions.  We then compare
our observations and analysis in this study to those of related work
and conclude with a summary of the key novel observations arising
from this study.


