%%
%% Template conclusion.tex
%%


\section{Summary}
\label{sec:summary}

In this thesis we have tested and compared different feature vectors across different exposures of size $k$. 
We have shown that \emph{User Interactions} in themselves are not predictive of user likes, however coupled with the 
likes exposure curve, they do show an improvement over our baselines as $k$ increases.

We have also shown the interesting result that \emph{User Preferences} are predictive of user likes in the base case of $k = 0$ and 
this trend continues over the likes exposure curve.

To answer the question initially proposed for this thesis, we have shown the feature vector which provides the highest predictiveness 
of user likes is the combined vector comprised of \emph{Traits}, \emph{Groups} and \emph{Pages}. These were the highest performing 
individual feature vectors and combined, represent the most predictive feature vector across our testing scope.

Which is the exciting novel insight examined by this thesis.

\section{Future Work}
\label{sec:fw}

Proposed future work can be summarised under the following points:

\begin{itemize}
\item \textbf{Increase size ranges}: Given our maximum test sizes for \emph{Groups} and \emph{Pages} of $1000$ this size could be 
increased to find the optimal testing range for each of our classifiers.
\item \textbf{Individual \emph{Traits} analysis}: During our \emph{Traits} analysis the feature vector was set to $1$ if the user and any user in 
the set of alters were part of the same \emph{Traits} group, it could be beneficial to do an individual analysis on each component of 
the \emph{Traits} data to find which individual elements of each \emph{Trait} are most predictive (similar to the analysis as done for \emph{Groups} and \emph{Pages}).
\item \textbf{Passive likes}: Given the Facebook model of allowing users to like but not dislike data, explicit dislike data can not be gleaned
from Facebook, which is hence why the LinkR active likes data was used for this evaluation. An approach could be developed which can predict 
whether a user will have seen an item (online timestamps, recent interactions with user) and can infer that if the user did not like the item then
they disliked it. This data set could then be applied to the testing methodology outlined above.
\item \textbf{Cold start}: Leaving out some subset of users when training our models, but including them during testing to explore their 
effects on results.
\item \textbf{General user set}: Such as the study done by ~\cite{jugand} which comprised of the entire active social network of 721 million users 
as of May 2011, applying these methods to a data set which is more indicative of the general Facebook user population could offer more generalisable 
results.
\item \textbf{Bayesian Model Averaging}: Weighting the most successful machine learning models under different feature sets and exposure curves 
to simulate a new combined classifier, which has learnt from the positive results of each individual classifier.
\end{itemize}

%%% Local Variables: 
%%% mode: latex
%%% TeX-master: "thesis"
%%% End: 
