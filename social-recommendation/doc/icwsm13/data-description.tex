
We built a Facebook App\footnote{Name and link omitted
for anonymity.} to collect information about users, 
their interactions and preferences.  
%data about Facebook users and their Facebook friends
%are collected through our Facebook App. 
Our dataset contains information about each App user, along with a
subset of information about their friends visible to the App.  The
data collection is performed with full permission from the user and in
accordance with an approved Ethics Protocol\footnote{Link omitted for
anonymity.}.

Over 119 users installed the Facebook App during the
evaluation period. From these core App users, the App has access to their
detailed Facebook profiles and their interactions with a total of
38,259 friends.

%interaction data.  
While we have complete interaction data for the App
users with their friends, and profile data (including
wall post data) for the App users and friends, we do not have complete
interactions for the App users' friends (unless they themselves are
App users).  Hence in the forthcoming analysis, we limit our
evaluation to App users for which we are assured to have full
interaction data.

Our App tracks app user's (and their friends') details and interactions
on Facebook.  Interactions that occur through wall posts provide a
rich variety of content and interaction data.  We distinguish four
Facebook items from wall posts: general posts (e.g., status updates,
activity updates such as new friends, and interactions such as the
user liked these pages), links, photos and videos. Four main
interactions on these items are permitted by Facebook: posting an item
to a friend's wall, commenting, liking, and tagging\footnote{Some
Facebook interaction features such as liking comments were introduced
after App user studies began and so are not tracked.}.  The App does
not track deletions of these items and interactions (e.g., unlike) for
performance reasons and we found very few deletions during an initial
testing stage.

We summarize relevant basic statistics of the data in Table~\ref{tab:interactions}-\ref{tab:interests} below.
The tables distinguish the data from the App users and from
all App users and friends. Table~\ref{tab:interactions}
summarizes the number of records for each item (row) and interaction (column)
combination. Table~\ref{tab:demographics} shows 
some demographics from user profiles.

We categorize facebook user-user and user-entities relationship into two broad categories ie \textbf{Interactions} and \textbf{Activities}.

\subsection{Interactions}
Interactions describes the communication between users based on modality, action and direction
\begin{itemize}
\item \textbf{Interaction Modality :}  User can interact with other users via
									 \textit{links, posts, photos} and \textit{videos}
\item \textbf{Interaction Action :}  A user $u$ can \textit{comment} or \textit{like} 
									user $v$'s item or \textit{tag} user $v$ on an 
									item(with one exception in Facebook that $u$ cannot tag a 
									link with users $v$ for the obvious reason) 
\item \textbf{Interaction Directionality :} Motivated by~\cite{saez2011high}, we can look
      								at \textit{incoming} and \textit{outgoing} interactions, i.e.,
      								if user $u$ comments on, tags, or likes user $v$'s item,
      								then $u$ is in the set of incoming interactions for $v$
      								and $v$ is in the set of outgoing interactions for $u$.
      								
\end{itemize}								

\subsection{Activities}
Activities describes the user interactions with non-user Facebook entities like groups, poges, favourite movies, favourite music etc.
\begin{itemize}
  %@SCOTT/LEXING following defination are taken from facebook blog. I am confused how to cite it %
  \item \textbf{Groups :} Facebook Groups are the place for small group communication and for people to share their common interests 
  						and express their opinion. Groups allow people to come together around a common cause, issue or activity to
  						organize, express objectives, discuss issues, post photos and share related content.
  \item \textbf{Pages :}  Facebook Pages enable public figures, businesses, organizations and other entities to create an authentic 
  						and public presence on Facebook. Facebook Pages are visible to everyone on the internet 
  						by default. Facebook user can connect with these Pages by becoming a fan and then receive their updates and interact with them.
  \item \textbf{Favourites :} Facebook facilitates a wide variety of user selected favourites (Activities, Favorite Athletes, Books, Interests, Movies, Music, Sports, Favorite Teams, Television ). These favourites allow
							a user to associate themselves with other people who share their same favourite tendencies.
\end{itemize} 
 
      							
 

\begin{table}
\centering
\begin{tabular}{|>{\small}l|>{\small}r|>{\small}r|>{\small}r|>{\small}r|}
\hline
\textbf{App Users} & \textbf{Posts} & \textbf{Tags} & \textbf{Comments} & \textbf{Likes} \\
\hline
\textbf{Wall} & 36,359 & 7,711 & 22,388 & 15,999 \\
\hline
\textbf{Link} & 5,304 & --- & 7,483 & 6,566 \\
\hline
\textbf{Photo} & 4,933 & 28,341 & 10,976 & 8,612 \\
\hline
\textbf{Video} & 245 & 2,525 & 1,970 & 843 \\
\hline
\hline
\textbf{App Users} & \textbf{Posts} & \textbf{Tags} & \textbf{Comments} & \textbf{Likes} \\
\textbf{and Friends} & & & & \\
\hline
\textbf{Wall} & 4,301,306 & 1,215,382 & 3,122,019 & 1,887,497 \\
\hline
\textbf{Link} & 678,612 & --- & 891,986 & 995,214 \\
\hline
\textbf{Photo} & 1,268,816 & 9,620,708 & 3,431,321 & 2,469,859 \\
\hline
\textbf{Video} & 59,244 & 904,604 & 486,677 & 332,619 \\
\hline
\end{tabular}
\caption{Number of records in Items and Interactions Tables. Rows are type of Facebook item and columns are type of Facebook interaction.}
\label{tab:interactions}
\end{table}


\begin{table}
\centering
\begin{tabular}{|>{\small}l|>{\small}r|>{\small}r|}
\hline
& \textbf{App Users} & \textbf{App User} \\
& & \textbf{and Friends} \\
\hline
Users & 119 & 38,378 \\
\hline
Male & 85 & 20,840 \\
\hline
Female & 33 & 17,032 \\
\hline
\end{tabular}
\caption{App user demographics \footnote{Note that some people don't share gender information in facebook}}
\label{tab:demographics}
\end{table}

\begin{table}
\centering
\begin{tabular}{|>{\small}l|>{\small}r|>{\small}r|}
\hline
& \textbf{App Users} & \textbf{App User} \\
& & \textbf{and Friends} \\
\hline
Groups & 3,469 & 373,608 \\
\hline
Page Likes & 10,771 & 825,452 \\
\hline
Favourites & 4,284 & 892,820\\
\hline
\end{tabular}
\caption{App user and friends activity statistics }
\label{tab:interests}
\end{table}

\begin{table}
\centering
\begin{tabular}{|>{\small}l|>{\small}r|>{\small}r|}\hline
&\textbf{Friend}  & \textbf{Non-Friend} \\
&\textbf{recommendation}  & \textbf{recommendation} \\
\hline
Like& 1392 & 1127 \\
\hline
Dislike& 895 & 2111\\
\hline
\end{tabular}
\caption{Dataset breakdown by friend/non-friend and like/dislike}
\end{table}


%\begin{table}
%\centering
%\begin{tabular}{|c|c|}
%\hline
% Favourite & \# items \\
%\hline
%        Activities  &     228 \\
%  FavoriteAthletes  &      57 \\
%             Books  &     126 \\
%         Interests  &      92 \\
%            Movies  &     360 \\
%             Music  &     671 \\
%            Sports  &      12 \\
%     FavoriteTeams  &      28 \\
%        Television  &     361 \\
%\hline
%\end{tabular}
%\caption{Favourite item counts}
%\end{table}
