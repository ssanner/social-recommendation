%!TEX root = document.tex

\label{sec:introduction}

% motivate socical recommendation 
Online social networks such as Facebook record a rich set of user
preferences (likes of links, posts, photos, videos), user traits,
interactions and activities (conversation streams, tagging, group
memberships, interests, personal history, and demographic data).  This
presents myriad new dimensions to the recommendation problem by making
available a rich labeled graph structure of social interactions and
content from which user preferences can be learned and new
recommendations can be made.

Most existing recommendation methods for social networks aggregate
this rich social information into a simple measure of user-to-user
interaction ~\cite{socinf,rrmf,sr,Noel2012NOF,lla,ste,sorec}.  But in
aggregating all of these interactions and common activities into a
\emph{single} strength of interaction, we
ask whether important preference information has been discarded?
Indeed, the point of departure for this work is the hypothesis that
different fine-grained interactions (e.g. commenting on a wall or
getting tagged in a video) and activities (e.g., being a member of a
university alumni group or a fan of a TV series) \emph{do} represent
different preferential {\em affinities} between users, and moreover
that effective {\em filtering} of this information (i.e., learning
which of these myriad fine-grained interactions and activities are 
informative) will lead to improved accuracy in social recommendation.

To quantitatively validate our hypotheses and evaluate the
informativeness of different fine-grained features for social
recommendation, we have built a Facebook App to collect detailed user
interaction and activity history available through the Facebook Graph
API along with user preferences solicited by the App on a daily basis.
Given this data, (1) we define a novel recommendation method called
{\em social affinity filtering (SAF)}, where we learn to predict
whether a user (ego) will like an item based on the surrogate item
preferences of others (alters) who share fine-grained interactions or
activities with the ego, and (2) we analyse the relative
informativeness of these fine-grained interaction and activity
features across a variety of dimensions.

In the four months that our App was active, we collected data for a
set of Facebook app users and their full interactions with 37,000+
friends along with 22 distinct types of interaction and users activity
for 3000+ groups, 4000+ favourites, and 10,000+ pages.  In subsequent
sections that outline our experimental methodology and results in
detail, we make the following critical observations:
\begin{itemize}
\item {\bf Overall performance:} We found that SAF significantly 
outperforms numerous state-of-the-art collaborative filtering and
social recommender systems by over 6\% in accuracy using
just \emph{page} (like) features.  
\item {\bf Privacy vs. performance:} 
Because the reluctance of a user to install an App increases with the
number of permissions requested, the above results suggest that an
SAF-based social recommendation App need only request permissions for
a user's \emph{page} likes in order to achieve state-of-the-art
recommendation accuracy.
\item {\bf Cold-start performance:}
Since SAF does not condition its predictions on individual
user IDs, we show that it exhibits strong \emph{cold-start}
performance for users \emph{without} expressed item preferences as long as those users
have interactions or shared activities with users who have
expressed item preferences.
\item {\bf Big data scalability:} We implement SAF as 
a simple linear classifier that
can be used in conjunction with a variety of classification methods (e.g., naive
Bayes, logistic regression, SVM) and online training algorithms
amenable to real-time, big data settings.
\item {\bf Interaction analysis:} 
Among \emph{interactions}, we found that those on videos are more
predictive than those on other content types (photos, post, link), and
that outgoing interactions (performed by the ego on the alter's
timeline) are more predictive than incoming ones (performed by alters
on the ego's timeline), although the level of exposure of an ego to an
alter's preferences is often more important than the directionality,
modality, or action underlying the interaction with the alters.
\item {\bf Activity analysis:}
The most predictive activity SAGs tend to have small memberships
indicating that these informative activities represent highly
specialised interests.  We also found features corresponding to
``long-tailed'' dynamic content (such as music and books) can be more
predictive than those with fewer choices that add little new content
over time (e.g. interests or sports).
\item {\bf Importance of social data beyond friends:} 
We found that \emph{groups}, \emph{pages}, and \emph{favourites} 
make for more informative SAGs than those defined by user-to-user
interactions.  This is likely because the former can be applied to
SAGs \emph{over the entire Facebook population} rather than just a
user's friends (where the available preference data is considerably
more sparse).
\item {\bf Social activity and item popularity vs. performance:} 
We analyse \emph{how many} shared activities are needed for
good performance and observe that increased activity membership correlates
with increased recommendation accuracy, but that excessive item popularity
coupled with high activity membership hurts the discriminative power
of SAF to make good recommendations.
\item {\bf Fine-grained vs. aggregate social data:}
Among activity features, a small subset proved to be much more informative
than the rest.  This suggests the value of
learning \emph{which} fine-grained features are predictive and sheds
doubt on the efficacy of existing social recommendation methods that
aggregate social information between two users into a single numerical
value.
\end{itemize}
Subsequent sections demonstrate these findings in detail.




