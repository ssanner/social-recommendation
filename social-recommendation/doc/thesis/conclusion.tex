%%
%% Template conclusion.tex
%%

\chapter{Conclusion}

\section{Summary}

In this thesis, we evaluated existing algorithms and proposed new
algorithms for social collaborative filtering via the task of link
recommendation on Facebook.

%\begin{itemize}
In Chapter~2, we outlined three main deficiencies in current social
collaborative filtering (SCF) techniques and proposed new techniques
in Chapters~4 and 5 to solve them; we review them here as follows:

\begin{itemize}
\item[(a)] {\bf Non-feature-based user similarity:} 
We extended existing social regularization and \emph{social spectral regularization} methods to incorporate \emph{user features} to learn user-user similarities in the latent space.
\item[(b)] {\bf Model direct user-user information diffusion:} 
We defined a new hybrid SCF method where we \emph{combined} the \emph{collaborative filtering (CF) matrix factorization (MF) objective} used by Matchbox~\cite{matchbox} with a \emph{linear content-based filtering (CBF) objective} used to model direct user-user information diffusion in the social network.
\item[(c)] {\bf Restricted common interests:}
We defined a new social co-preference regularization method that \emph{learns from pairs of user preferences} over the same item to learn \emph{user similarities in specific areas} --- a contrast to previous methods that typically enforce global user similarity when regularizing.
\end{itemize}

Having evaluated existing baselines (with minor extensions) in
Chapter~4 and evaluating these new algorithms in Chapter~5 in live
online user trials with over 100 Facebook App users and data for over
30,000 unique Facebook users, we summarize the main results of the
thesis:

\begin{itemize}
\item Our extensions to social regularization in Chapter~4 and
Chapter~5 proved to be very effective methods for SCF, and
outperformed all other algorithms in the first Facebook user
evaluation trial.  This was reflected in the live user trial, user
survey, and offline passive experimental results.  However, our
socially regularized SCF MF extension in Chapter 4 could still be
improved by changing to a spectral approach, and this further
extension in Chapter~5 generally outperformed the SCF socially
regularized extension of Chapter~4.  The take-home point is that a very 
useful form of social regularization appears to be \emph{social spectral
regularization} as proposed in Chapter~5.

\item Learning direct user-user information diffusion models (i.e.,
how much one user likes links posted by another) can result in
improved SCF algorithms in comparison to standard MF methods.  To this
end, our \emph{social hybrid} algorithm which uses this information
outperformed all other SCF methods when recommending \emph{friend links} as
shown in Chapter~5.

\item Friend interaction information coming from social regularization
is more useful than the implicit co-likes information of co-preference
regularization. However, when there is no social interaction
information available (as in the case of recommending \emph{non-friend
links}), learning this implicit co-likes information outperforms plain
CF methods as evidenced by the relative success of the
\emph{co-preference regularization} algorithm as demonstrated in
Chapter~5.

\item Knowing what offline metrics for measuring SCF performance
correlate with human preferences from live trials is crucial for
efficient evaluation of SCF algorithms and also quite useful for
offline algorithm tuning.  In general, as most strongly evidenced in
Chapter~4, it appears that evaluating \emph{mean average precision}
(MAP) when training on data that includes both passively inferred
dislikes and explicit negative preference feedback (i.e., explicit
dislikes indicated via the Facebook App), but evaluating MAP only with
explicit feedback (no inferred dislikes) seems to be a train/test
evaluation approach and evaluation metric that correlates with human
survey feedback.

\end{itemize}

Hence, this thesis has made a number of substantial contributions to
SCF recommendation systems and has helped advance methods for
SCF system evaluation.  

\section{Future Work}

This work just represents the tip of the iceberg in different
improvements that SCF can make over more traditional non-social CF
methods.  Here we identify a number of additional future extensions
that can potentially further improve the proposed algorithms in this thesis:

\begin{itemize}
\item We used only a subset of the possible feature data in Facebook
and in the links themselves. Extending the social recommenders to
handle more of the rich information that is available may result in
better performance for the SCF algorithms. One critical feature that
would have been useful is including a $\mathit{genre}$ feature in the
links (e.g., indicating whether the link represented a blog, news,
video, etc.)  to provide a fine-grained model of which types of links
that users prefers to receive.  This additional information would have
likely prevented a number of observed dislikes from users regarding
specific genres of links, e.g., those that do not listen to
music much and hence do not care about links to music videos --- even
if these links are otherwise liked by friends and very popular.
\item Enforcing diversity in the recommended links would prevent
redundant links about the same topic being recommended again and
again. This is especially useful when an unusual event happens like
the death of Steve Jobs and the ensuing massive amount of Steve Jobs
related links that flooded Facebook.  While users may like to see a
few links on the topic, their interest in similar links decreases over
time and diversity in recommendations could help address this
saturation effect.
\item Another future direction this work can go to is to incorporate
active learning in the algorithms.  This would ensure that the SCF
algorithm did not exploit the learned preferences too much and made an
active effort to discover better link preferences that are available.
\item Probably the biggest assumption we have made in our
implementations is how we inferred the implicit dislikes of users in
the Facebook data.  A better and method of inferring implicit dislikes
will give a big boost to the SCF algorithms. As evidenced by the
results of training on ACTIVE and UNION data, having a more accurate
list of likes and dislikes greatly improves the performance of the SCF
algorithms.
\item Finally, there may be a better metric than MAP for offline
evaluation that more accurately correlates with live human
preference and research should continue to evaluate a variety
of existing SCF evaluation metrics in order to identify what
offline evaluation metrics correlate with human judgments of
algorithm performance.
\end{itemize}

While there are many exciting extensions of this work possible
as outlined above, 
this thesis represents a critical step forward in SCF
algorithms based on top-performing MF methods and their ability to
fully exploit the breadth of information available on social networks
to achieve state-of-the-art link recommendation.

%%% Local Variables: 
%%% mode: latex
%%% TeX-master: "thesis"
%%% End: 
