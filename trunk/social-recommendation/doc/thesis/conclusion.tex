%%
%% Template conclusion.tex
%%

\chapter{Conclusion}

\section{Summary}

We summarize the results and contributions of this paper as follows:

\begin{itemize}
\item{
We outlined three main deficiencies in current SCF techniques and discuss how we solve them:

\begin{itemize}
\item[(a)] {\bf Non-feature-based user similarity:} 
We extended existing social regularization and social spectral regularization methods to incorporate user features to learn user-user similarities in the latent space.
\item[(b)] {\bf Model direct user-user information diffusion:} 
We defined a new hybrid SCF method which a linear CBF function with a CF MF method to to incorporate user-user information diffusion into the model. 
\item[(c)] {\bf Restricted common interests:}
We defined a new social co-preference method that learns user like or dislike similarities only for specific areas.
\end{itemize}
}

\item{Our extension to social regularization is a very effective method for SCF, and beat all other algorithms in the first trial. This was reflected in the live user trial, user survey, and offline passive experiments. However, social regularization can still be improved to address other deficiencies outlines in this thesis.}

\item{Learning user-user information diffusion models can result in better SCF algorithms. Our Social Hybrid algorithm which uses this information outperformed all other SCF methods when recommending friend links.}

\item {Friend interaction information coming from social regularization is more useful than the implicit co-likes information of co-preference regularization. However  when there is no social interaction information available, learning this implicit co-likes information is better than plain CF methods.}

\end{itemize}

\section{Future Work}
\begin{itemize}
\item{We used only a subset of the possible feature data in Facebook and in the links themselves. Extending the recommenders to handle more of the rich information that is available may result in better performance for the SCF algorithms. One feature that would have been useful is including a $genre$ feature in the links (blog, news, video, etc.) to  fine-grain which types of links that users prefers to receive. This would have prevented a number of dislikes that from users that do not listen to music much and kept getting recommended music videos from YouTube. A specific genre like ``music video" in the links would have allowed the SCF algorithms to learn these types of user preferences.}
\item{Related to having more features like the genre of the links, enforcing diversity in the recommended links would prevent redundant  links about the same topic being recommended again and again. This is especially useful when an unusual event happens like the death of Steve Jobs and the ensuing massive amount of links about it being posted on Facebook by other users.}
\item{Another future direction this work can go to is to incorporate active learning in the algorithms. This would ensure that the SCF algorithm do not exploit the learned preferences too much and even discover better link preferences that are available.}
\item{Probably the biggest assumption we have made in our implementations is how we inferred the implicit dislikes of users in the Facebook data. A better and method of inferring implicit dislikes will give a big boost to the SCF algorithms. As evidenced by the results of training on ACTIVE and UNION data, having a more accurate list of likes and dislikes greatly improves the performance of the SCF algorithms. }
\item{Finally, there may be a better metric than MAP for offline testing that more accurately correlates with live human preference. What that metric is is still an open question.}
\end{itemize}

%%% Local Variables: 
%%% mode: latex
%%% TeX-master: "thesis"
%%% End: 
